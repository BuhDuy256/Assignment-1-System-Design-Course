\section{Abstract Factory (Creational)}

\begin{frame}
  \begin{center}
    \Huge\textbf{Abstract Factory}\\[0.5em]
    \Large\textcolor{myblue}{Creational Pattern}
  \end{center}
\end{frame}

% ============================================================
% SLIDE 1 — BỐI CẢNH: Khi nào vấn đề xuất hiện?
% ============================================================
\begin{frame}[t]{Abstract Factory: Bối cảnh thực tế}
  \small 
  \textbf{Tình huống phổ biến:}
  \begin{itemize}
    \setlength{\itemsep}{0pt} % Loại bỏ khoảng cách thừa giữa các dòng
    \item Hệ thống cần tạo ra các \textbf{họ sản phẩm liên quan}.
    \item Các sản phẩm phải đảm bảo tính \textbf{tương thích} tuyệt đối.
  \end{itemize}

  \vspace{0.2em} % Thu hẹp khoảng cách giữa các khối
  \textbf{Ví dụ thực tế:}
  \begin{itemize}
    \setlength{\itemsep}{0pt}
    \item \textcolor{myblue}{Furniture Shop}: Ghế, bàn, sofa theo bộ (\textit{Modern}, \textit{Victorian}).
    \item \textcolor{myblue}{Cross-Platform UI}: Toolkit (Button, Input) đồng bộ theo HĐH.
    \item \textcolor{myblue}{Game Themes}: Bộ quái vật, vũ khí theo chủ đề \textit{Fantasy}.
  \end{itemize}

  \vspace{0.3em}
  \centering
  \textcolor{myred}{\textbf{Mấu chốt:}} Tránh "râu ông nọ cắm cằm bà kia" (vd: ghế \textit{Modern} trong bộ bàn \textit{Victorian}).
\end{frame}

% ============================================================
% SLIDE 2 — NỖI ĐAU: Code không có pattern
% ============================================================
\begin{frame}[fragile]{Abstract Factory: Nỗi đau khi thiếu Pattern}
  \small
  \begin{columns}[T]
    \begin{column}{0.55\textwidth}
      \textbf{Cách làm ``Thủ công'':}\\[0.2em]
      {\scriptsize Kiểm tra điều kiện ở mọi nơi}
      \vspace{0.2em}
      \begin{minipage}[t]{\textwidth}
      \fontsize{7pt}{8pt}\selectfont
      \begin{verbatim}
// Code rải rác khắp nơi
if (style == "Modern") {
    chair = new ModernChair();
    sofa = new ModernSofa();
} else if (style == "Victorian") {
    chair = new VictorianChair();
    sofa = new VictorianSofa();
}
      \end{verbatim}
      \end{minipage}
    \end{column}
    \begin{column}{0.42\textwidth}
      \textbf{Hậu quả:}
      \begin{itemize}
        \item \textcolor{myred}{Inconsistency}: Dễ chọn nhầm sản phẩm không cùng họ.
        \item \textcolor{myred}{Rigidity}: Thêm một phong cách mới yêu cầu sửa `if/else` ở tất cả các module.
        \item \textcolor{myred}{Tight Coupling}: Client phụ thuộc vào hàng tá concrete classes của sản phẩm.
      \end{itemize}
    \end{column}
  \end{columns}
\end{frame}

% ============================================================
% SLIDE 3 — INSIGHT CỐT LÕI
% ============================================================
\begin{frame}{Abstract Factory: Insight cốt lõi}
  \small
  \begin{center}
    \textbf{\textcolor{myblue}{Factory of Factories}} % Bỏ \Large để tiết kiệm chỗ
  \end{center}
  \vspace{-0.5em} % Thu hẹp khoảng cách
  \textbf{Ý tưởng then chốt:}
  \begin{enumerate}
    \item \textbf{Interface họ}: Một đối tượng chứa nhiều Factory Methods cho cả họ sản phẩm.
    \item \textbf{Composition}: Client ủy quyền việc tạo cho đối tượng Factory.
    \item \textbf{Invariant}: Cam kết sản phẩm tạo ra luôn cùng biến thể.
  \end{enumerate}

  \textbf{Nguyên lý thiết kế:}
  \begin{itemize}
    \item \textcolor{mygreen}{\textbf{DIP}}: Chỉ làm việc với Abstract interfaces.
    \item \textcolor{mygreen}{\textbf{OCP}}: Thêm họ sản phẩm mới mà không sửa Client.
  \end{itemize}
\end{frame}

% ============================================================
% SLIDE 4 — CẤU TRÚC & CÁCH HOẠT ĐỘNG
% ============================================================
\begin{frame}[fragile]{Abstract Factory: Cách hoạt động}
  \small
  \textbf{Các vai trò trong pattern:}
  \begin{itemize}
    \item \textcolor{myblue}{\textbf{Abstract Factory}}: Khai báo các phương thức tạo từng loại sản phẩm trừu tượng.
    \item \textcolor{mygreen}{\textbf{Concrete Factory}}: Triển khai việc tạo các sản phẩm cụ thể cho một biến thể duy nhất.
    \item \textcolor{mypurple}{\textbf{Abstract Products}}: Các interface cho từng loại sản phẩm trong họ.
  \end{itemize}

  \textbf{Flow thực thi:}
  \begin{enumerate}
    \item \textbf{Cấu hình}: Lúc khởi tạo, app chọn một Concrete Factory (vd: \texttt{ModernFactory}).
    \item \textbf{Truyền Factory}: Client nhận factory này thông qua constructor/setter.
    \item \textbf{Tạo bộ sản phẩm}: Client gọi \texttt{factory.createChair()} và \texttt{factory.createSofa()}.
    \item \textbf{Vận hành}: Các sản phẩm trả về tự động tương thích vì chúng cùng xuất phát từ một factory biến thể.
  \end{enumerate}

  \vspace{0.2em}
  \textcolor{mygreen}{\textbf{Kết quả:}} Client code hoàn toàn sạch bóng các từ khóa \texttt{new} cho sản phẩm cụ thể.
\end{frame}

% ============================================================
% SLIDE 5 — TRADE-OFFS
% ============================================================
\begin{frame}{Abstract Factory: Được và Mất}
  \small
  \begin{columns}[T]
    \begin{column}{0.48\textwidth}
      \textcolor{mygreen}{\textbf{Được gì:}}
      \begin{itemize}
        \item \textbf{Consistency}: Đảm bảo các sản phẩm luôn đi kèm đúng bộ.
        \item \textbf{Decoupling}: Client và Concrete Products không biết nhau.
        \item \textbf{SRP}: Logic tạo các họ sản phẩm được tập trung hóa.
      \end{itemize}
    \end{column}
    \begin{column}{0.48\textwidth}
      \textcolor{myred}{\textbf{Mất gì:}}
      \begin{itemize}
        \item \textbf{Complexity}: Quá nhiều interface và class mới làm code trở nên cồng kềnh.
        \item \textbf{Hard to Extend}: Thêm một \textit{loại} sản phẩm mới (vd: CoffeeTable) vào họ yêu cầu sửa interface Factory gốc.
      \end{itemize}
    \end{column}
  \end{columns}

  \vspace{0.5em}
  \centering
  \textcolor{myorange}{\textbf{Lưu ý:}} Thường tiến hóa từ \textbf{Factory Method} khi số lượng họ sản phẩm bắt đầu tăng lên.
\end{frame}

% ================================
% SLIDE 6 - SO SÁNH FACTORY METHOD vs ABSTRACT FACTORY
% ================================

\begin{frame}{Factory Method vs Abstract Factory}

\textbf{Giống nhau:}
Creational Pattern • Ẩn việc \texttt{new} khỏi client • Tuân theo OCP

\vspace{0.3cm}

\textbf{Khác nhau cốt lõi}

\small
\setlength{\tabcolsep}{4pt} % giảm khoảng cách cột

\begin{center}
\begin{tabular}{|l|l|l|}
\hline
 & \textbf{Factory Method} & \textbf{Abstract Factory} \\
\hline
Tạo gì? & Một Product & Một họ Product \\
\hline
Mức độ & Class override & Object composition \\
\hline
Mục tiêu & Chọn implementation & Chọn ecosystem \\
\hline
Scale & Nhỏ, đơn giản & Nhiều product đi cùng \\
\hline
Ví dụ & 1 loại Document & Cả bộ UI Windows/Mac \\
\hline
\end{tabular}
\end{center}

\normalsize
\vspace{0.3cm}

\begin{block}{Tóm tắt}
Factory Method → tạo \textbf{một} sản phẩm \\
Abstract Factory → tạo \textbf{cả một hệ sinh thái} sản phẩm.
\end{block}

\end{frame}

% ============================================================
% SLIDE 7 — KHI NÀO DÙNG?
% ============================================================
\begin{frame}{Abstract Factory: Khi nào dùng?}
  \small
  \textcolor{mygreen}{\textbf{Nên dùng khi:}}
  \begin{itemize}
    \item Code cần làm việc với nhiều họ sản phẩm mà không muốn phụ thuộc vào class cụ thể.
    \item Muốn cưỡng chế việc sử dụng đồng bộ các đối tượng liên quan.
  \end{itemize}

  \vspace{0.3em}
  \textcolor{myred}{\textbf{Tránh dùng khi:}}
  \begin{itemize}
    \item Các lớp sản phẩm cụ thể \textbf{không bao giờ thay đổi} hoặc chỉ có một biến thể duy nhất.
    \item Bài toán đơn giản, việc tách class làm tăng Cognitive Load vô ích.
  \end{itemize}
\end{frame}