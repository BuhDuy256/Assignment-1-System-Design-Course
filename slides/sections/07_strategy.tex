\section{Strategy (Behavioral)}

\begin{frame}
  \begin{center}
    \Huge\textbf{Strategy}\\[0.5em]
    \Large\textcolor{myorange}{Behavioral Pattern}
  \end{center}
\end{frame}

% ============================================================
% SLIDE 1 — BỐI CẢNH
% ============================================================
\begin{frame}{Strategy: Bối cảnh thực tế}
  \textbf{Vấn đề:}
  \begin{itemize}
    \item Bạn có một tác vụ cụ thể cần thực hiện.
    \item Có \textbf{nhiều cách} (thuật toán) để giải quyết tác vụ đó.
    \item Cần chuyển đổi linh hoạt giữa các cách này (thậm chí lúc runtime).
  \end{itemize}

  \vspace{1em}
  \textbf{Ví dụ dễ hiểu:}
  \begin{itemize}
    \item \textcolor{myblue}{Bản đồ (Google Maps):}
      Đi từ A đến B $\rightarrow$ Có thể chọn: \textit{Xe hơi, Xe máy, Xe buýt, Đi bộ}.\\
      (Mục tiêu giống nhau, nhưng thuật toán tìm đường khác nhau).
    \item \textcolor{myblue}{Thanh toán:}
      Mua hàng $\rightarrow$ Có thể chọn: \textit{Thẻ tín dụng, Momo, PayPal, Tiền mặt}.
    \item \textcolor{myblue}{Sắp xếp:}
      List danh sách $\rightarrow$ Chọn: \textit{QuickSort, MergeSort, BubbleSort}.
  \end{itemize}

  \vspace{0.5em}
  \centering
  \textcolor{myred}{\textbf{Mấu chốt:}} Cùng một mục đích, nhưng cách làm (chiến thuật) khác nhau.
\end{frame}

% ============================================================
% SLIDE 2 — NỖI ĐAU (BAD SMELL)
% ============================================================
\begin{frame}[fragile]{Strategy: Code xấu khi thiếu Pattern}
  \begin{columns}[T]
    \begin{column}{0.48\textwidth}
      \textbf{Cách làm ``Ngây thơ'':}
      {\footnotesize Dồn hết logic vào một hàm khổng lồ.}
      
      \vspace{0.2em}
      \scriptsize
      \begin{verbatim}
class Navigator {
  void buildRoute(String type) {
    if (type.equals("ROAD")) {
      // 100 dòng code tìm đường bộ
      // Xử lý kẹt xe, đèn đỏ...
    } 
    else if (type.equals("WALK")) {
      // 100 dòng code đường đi bộ
      // Xử lý vỉa hè, công viên...
    }
    // else if ("PUBLIC_TRANSIT")...
  }
}
      \end{verbatim}
    \end{column}
    \begin{column}{0.48\textwidth}
      \textbf{Hậu quả:}
      \begin{itemize}
        \item \textcolor{myred}{Class Khổng Lồ}: Khó đọc, khó bảo trì.
        \item \textcolor{myred}{Vi phạm OCP}: Muốn thêm loại phương tiện mới $\rightarrow$ Phải sửa class cũ.
        \item \textcolor{myred}{Khó test}: Lỗi ở thuật toán "Đi bộ" có thể làm hỏng cả chức năng "Đi xe".
      \end{itemize}
    \end{column}
  \end{columns}
\end{frame}

% ============================================================
% SLIDE 3 — GIẢI PHÁP & INSIGHT
% ============================================================
\begin{frame}{Strategy: Insight cốt lõi}
  \begin{center}
    \Large\textcolor{myblue}{\textbf{Encapsulate Algorithms}}
  \end{center}

  \vspace{0.5em}
  \textbf{Ý tưởng then chốt:}
  \begin{enumerate}
    \item Tách từng thuật toán ra khỏi class chính.
    \item Đưa mỗi thuật toán vào một class riêng gọi là \textbf{Strategy}.
    \item Các Strategy này phải tuân thủ cùng một \textbf{Interface}.
    \item Class chính (\textit{Context}) chỉ giữ một tham chiếu đến Interface này.
  \end{enumerate}

  \vspace{0.5em}
  \textbf{Lợi ích:}
  \begin{itemize}
    \item \textcolor{mygreen}{\textbf{Context}}: Không cần biết chi tiết thuật toán, chỉ cần gọi hàm \texttt{execute()}.
    \item \textcolor{mygreen}{\textbf{Linh hoạt}}: Có thể tráo đổi thuật toán ngay khi chương trình đang chạy.
    \item \textcolor{mygreen}{\textbf{Độc lập}}: Các thuật toán phát triển riêng biệt, không ảnh hưởng nhau.
  \end{itemize}
\end{frame}

% ============================================================
% SLIDE 4 — CODE VÍ DỤ
% ============================================================
\begin{frame}[fragile]{Strategy: Ví dụ minh họa}
  \small
  \textbf{Bài toán:} Thanh toán đơn hàng với nhiều phương thức khác nhau.
  
  \vspace{0.5em}
  \scriptsize
  \begin{verbatim}
// 1. Strategy Interface
interface PaymentStrategy { void pay(int amount); }

// 2. Concrete Strategies (Các thuật toán cụ thể)
class CreditCard implements PaymentStrategy {
    public void pay(int amount) { print("Quẹt thẻ: " + amount); }
}

class PayPal implements PaymentStrategy {
    public void pay(int amount) { print("Trừ ví PayPal: " + amount); }
}

// 3. Context (Lớp ngữ cảnh sử dụng)
class Order {
    private PaymentStrategy strategy;

    public void setStrategy(PaymentStrategy s) { this.strategy = s; }

    public void checkout(int money) {
        // Ủy quyền việc xử lý cho Strategy
        strategy.pay(money); 
    }
}
  \end{verbatim}
  \centering
  \small \textcolor{mygreen}{Client có thể đổi từ CreditCard sang PayPal dễ dàng.}
\end{frame}

% ============================================================
% SLIDE 5 — TRADE-OFFS
% ============================================================
\begin{frame}{Strategy: Được và Mất}
  \begin{columns}[T]
    \begin{column}{0.48\textwidth}
      \textcolor{mygreen}{\textbf{Ưu điểm:}}
      \begin{itemize}
        \item \textbf{Clean Code:} Loại bỏ các khối `if/else` hoặc `switch` phức tạp.
        \item \textbf{Open/Closed Principle:} Thêm thuật toán mới không cần sửa code cũ.
        \item \textbf{Tách biệt (Isolation):} Logic nghiệp vụ tách rời khỏi chi tiết triển khai thuật toán.
      \end{itemize}
    \end{column}
    \begin{column}{0.48\textwidth}
      \textcolor{myred}{\textbf{Nhược điểm:}}
      \begin{itemize}
        \item \textbf{Số lượng Class tăng:} Mỗi thuật toán là một file class mới.
        \item \textbf{Client phải hiểu:} Người dùng (Client) phải biết sự khác nhau giữa các Strategy để chọn cho đúng.
      \end{itemize}
    \end{column}
  \end{columns}

  \vspace{1em}
  \centering
  \fcolorbox{myorange}{white}{
    \begin{minipage}{0.9\textwidth}
      \centering \small
      \textbf{Lưu ý:} Các Strategy nên là \textit{stateless} (không lưu trạng thái) để tránh lỗi khi dùng chung.
    \end{minipage}
  }
\end{frame}

% ============================================================
% SLIDE 6 — KHI NÀO DÙNG?
% ============================================================
\begin{frame}{Strategy: Khi nào dùng?}
  \begin{columns}[T]
    \begin{column}{0.5\textwidth}
      \textcolor{mygreen}{\textbf{Nên dùng:}}
      \begin{itemize}
        \item Khi muốn sử dụng các biến thể khác nhau của một thuật toán.
        \item Khi có quá nhiều điều kiện `if/else` chỉ để chọn cách xử lý.
        \item Khi muốn giấu đi sự phức tạp của thuật toán khỏi người dùng.
      \end{itemize}
    \end{column}
    \begin{column}{0.5\textwidth}
      \textcolor{myred}{\textbf{Không nên dùng:}}
      \begin{itemize}
        \item Nếu chỉ có 1-2 thuật toán và chúng hiếm khi thay đổi.
        \item Logic quá đơn giản, việc tách class làm code rối thêm (Over-engineering).
      \end{itemize}
    \end{column}
  \end{columns}

  \vspace{1em}
  \centering
  \textbf{Dấu hiệu nhận biết:}
  \textit{``Cùng input đầu vào, nhưng cần output/cách xử lý khác nhau tùy hoàn cảnh.''}
\end{frame}