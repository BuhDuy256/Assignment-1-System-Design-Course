\section{Factory Method (Creational)}

\begin{frame}
  \begin{center}
    \Huge\textbf{Factory Method}\\[0.5em]
    \Large\textcolor{myblue}{Creational Pattern}
  \end{center}
\end{frame}

% ============================================================
% SLIDE 1 — BỐI CẢNH
% ============================================================
\begin{frame}{Factory Method: Bối cảnh thực tế}
  \small % Giảm kích thước chữ toàn slide để tránh tràn
  \textbf{Tình huống phổ biến:}
  \begin{itemize}
    \item Hệ thống cần tạo đối tượng (Product) nhưng chưa biết chính xác loại cụ thể lúc thiết kế.
    \item Việc khởi tạo phụ thuộc vào cấu hình hoặc logic tại runtime.
  \end{itemize}

  \textbf{Ví dụ thực tế:}
  \begin{itemize}
    \item \textcolor{myblue}{Logistics}: Ban đầu chỉ có \textit{Truck}, sau đó cần mở rộng thêm \textit{Ship}, \textit{Train}.
    \item \textcolor{myblue}{UI Framework}: Nút bấm thay đổi theo hệ điều hành.
    \item \textcolor{myblue}{Converters}: Xuất file sang PDF, Word, hoặc HTML.
  \end{itemize}

  \vspace{0.5em}
  \centering
  \textcolor{myred}{\textbf{Mấu chốt:}} Phụ thuộc vào \textbf{Concrete Class} làm hệ thống bị "cứng", khó mở rộng mà không sửa code cũ.
\end{frame}

% ============================================================
% SLIDE 2 — NỖI ĐAU
% ============================================================
\begin{frame}[fragile]{Factory Method: Nỗi đau khi thiếu Pattern}
  \begin{columns}[T]
    \begin{column}{0.52\textwidth}
      \textbf{Cách làm ``Ngây thơ'':}\\[0.2em]
      {\scriptsize Dùng \texttt{new} trực tiếp và switch-case}
      
      \vspace{0.3em}
      \begin{minipage}[t]{\textwidth}
      \fontsize{7pt}{8pt}\selectfont
      \begin{verbatim}
class Logistics {
  void planDelivery(String type) {
    Transport t;
    if (type == "Truck") 
        t = new Truck();
    else if (type == "Ship") 
        t = new Ship();
    t.deliver();
  }
}
      \end{verbatim}
      \end{minipage}
    \end{column}
    \begin{column}{0.44\textwidth}
      \small
      \textbf{Hậu quả:}
      \begin{itemize}
        \item \textcolor{myred}{Tight Coupling}: Phụ thuộc trực tiếp vào class cụ thể.
        \item \textcolor{myred}{Vi phạm OCP}: Thêm phương tiện mới $=$ Sửa logic cốt lõi.
        \item \textcolor{myred}{Khó Unit Test}: Không thể mock các class bị "hard-code".
      \end{itemize}
    \end{column}
  \end{columns}
\end{frame}

% ============================================================
% SLIDE 3 — INSIGHT CỐT LÕI (Slide bị lỗi trong hình)
% ============================================================
\begin{frame}{Factory Method: Insight cốt lõi}
  \small % Khắc phục lỗi tràn footer bằng cách dùng chữ nhỏ hơn
  \begin{center}
    \large\textcolor{myblue}{\textbf{Virtualize Object Creation}}
  \end{center}

  \textbf{Ý tưởng then chốt:}
  \begin{enumerate}
    \item \textbf{Đóng gói sự biến thiên}: Tách biệt logic \textit{sử dụng} và \textit{tạo} đối tượng.
    \item \textbf{Ủy quyền cho Subclass}: Class con quyết định \textit{Product} nào sẽ được tạo.
    \item \textbf{Lập trình với Interface}: Client không quan tâm Concrete class là gì.
  \end{enumerate}

  \vspace{0.2em}
  \textbf{Nguyên lý thiết kế:}
  \begin{itemize}
    \item \textcolor{mygreen}{\textbf{DIP}}: Phụ thuộc vào abstraction, không phụ thuộc vào concrete class.
    \item \textcolor{mygreen}{\textbf{Encapsulation of Variation}}: Biến thiên về loại sản phẩm được khóa sau Factory Method.
  \end{itemize}
\end{frame}

% ============================================================
% SLIDE 4 — CẤU TRÚC & CÁCH HOẠT ĐỘNG
% ============================================================
\begin{frame}[fragile]{Factory Method: Cách hoạt động}
  \small
  \textbf{Các vai trò trong pattern:}
  \begin{itemize}
    \item \textcolor{myblue}{\textbf{Product}}: Interface quy định các khả năng của sản phẩm.
    \item \textcolor{mygreen}{\textbf{Creator}}: Lớp chứa logic nghiệp vụ, khai báo Factory Method.
    \item \textcolor{mypurple}{\textbf{ConcreteCreator}}: Thực thi Factory Method để trả về sản phẩm cụ thể.
  \end{itemize}

  \vspace{0.3em}
  \textbf{Flow thực thi:}

  \vspace{0.2em}
  \begin{minipage}[t]{\textwidth}
  \fontsize{7pt}{8pt}\selectfont
  \begin{verbatim}
  // Client chỉ làm việc với lớp trừu tượng
  Logistics app = new SeaLogistics(); 
  app.planDelivery(); 
  // Inside planDelivery: 
  //   Transport t = createTransport(); // Lấy "Ship" tại runtime
  //   t.deliver();
  \end{verbatim}
  \end{minipage}

  \vspace{0.2em}
  \textcolor{mygreen}{\textbf{Kết quả:}} Logic nghiệp vụ không hề biết mình đang dùng \textit{Truck} hay \textit{Ship}.
\end{frame}

% ============================================================
% SLIDE 5 — TRADE-OFFS
% ============================================================
\begin{frame}{Factory Method: Được và Mất}
  \small
  \begin{columns}[T]
    \begin{column}{0.48\textwidth}
      \textcolor{mygreen}{\textbf{Được gì:}}
      \begin{itemize}
        \item \textbf{Loose Coupling}: Tránh phụ thuộc cứng.
        \item \textbf{Tuân thủ SRP}: Gom logic tạo object về một nơi.
        \item \textbf{Tuân thủ OCP}: Mở rộng dễ dàng.
      \end{itemize}
    \end{column}
    \begin{column}{0.48\textwidth}
      \textcolor{myred}{\textbf{Mất gì:}}
      \begin{itemize}
        \item \textbf{Class Explosion}: Tăng số lượng class con.
        \item \textbf{Phức tạp hóa}: Cần nhiều lớp trung gian hơn.
      \end{itemize}
    \end{column}
  \end{columns}

  \vspace{0.5em}
  \centering
  \textcolor{myorange}{\textbf{Lưu ý:}} Phù hợp khi logic tạo đối tượng phức tạp và cần mở rộng thường xuyên.
\end{frame}

% ============================================================
% SLIDE 6 — KHI NÀO DÙNG?
% ============================================================
\begin{frame}{Factory Method: Khi nào dùng? Khi nào là thừa?}
  \small
  \textcolor{mygreen}{\textbf{Nên dùng khi:}}
  \begin{itemize}
    \item Cần cho phép người dùng framework mở rộng thành phần nội bộ.
    \item Muốn tái sử dụng đối tượng thay vì tạo mới liên tục (caching).
  \end{itemize}

  \vspace{0.3em}
  \textcolor{myred}{\textbf{Over-engineering khi:}}
  \begin{itemize}
    \item Số lượng sản phẩm là \textbf{cố định} và rất ít
    \item Việc tạo object quá đơn giản, không có kế hoạch mở rộng.
  \end{itemize}

  \vspace{0.5em}
  \centering
  \fcolorbox{myorange}{white}{
    \begin{minipage}{0.9\textwidth}
      \centering \scriptsize
      \textbf{Câu hỏi:} Nếu loại bỏ pattern này, business rule nào sẽ bị phá đầu tiên?
    \end{minipage}
  }
\end{frame}
