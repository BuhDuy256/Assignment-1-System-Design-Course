\section{Tổng quan về Design Patterns}

% --- SLIDE 1: Vấn đề thực tế khi KHÔNG dùng Design Patterns ---
\begin{frame}{Khi không có Design Patterns...}
  \begin{itemize}
    \item \textbf{Tight coupling} --- thay đổi 1 class ảnh hưởng hàng loạt class khác
    \item \textbf{Code duplication} --- cùng logic lặp lại ở nhiều nơi
    \item \textbf{Khó mở rộng} --- thêm feature mới = viết lại cả module
    \item \textbf{Bảo trì tốn kém} --- fix bug chỗ này, sinh bug chỗ khác
    \item \textbf{Giao tiếp khó khăn} --- mỗi dev giải quyết cùng bài toán theo cách khác nhau
  \end{itemize}
\end{frame}

% --- SLIDE 2: Hậu quả trong dự án thực tế ---
\begin{frame}{Hậu quả trong thực tế}
  \begin{alertblock}{Dự án phần mềm thất bại vì...}
    \begin{itemize}
      \item Hệ thống ``cứng'' --- không thích ứng khi requirement thay đổi
      \item Technical debt tích lũy --- càng sửa càng rối
      \item Onboarding chậm --- dev mới không hiểu kiến trúc
      \item Refactor = viết lại từ đầu --- vì code quá phụ thuộc lẫn nhau
    \end{itemize}
  \end{alertblock}
  
  \vspace{0.5em}
  \centering
  \textit{$\Rightarrow$ Những vấn đề này đã được giải quyết từ lâu...}
\end{frame}

% --- SLIDE 3: Design Patterns giải quyết gì? ---
\begin{frame}{Design Patterns giải quyết những gì?}
  \begin{block}{Giải pháp đã được kiểm chứng}
    \begin{itemize}
      \item \textbf{Giảm coupling} --- các thành phần giao tiếp qua abstraction
      \item \textbf{Tăng extensibility} --- thêm mới mà không sửa code cũ (OCP)
      \item \textbf{Tái sử dụng} --- giải pháp template, áp dụng cho nhiều bài toán
      \item \textbf{Chuẩn hóa giao tiếp} --- ``dùng Observer pattern'' $>$ giải thích 20 dòng
    \end{itemize}
  \end{block}
  
  \vspace{0.5em}
  \centering
  \textit{Không phải code cụ thể --- mà là \textbf{cách tư duy thiết kế}.}
\end{frame}

% --- SLIDE 4: Phân loại 3 nhóm ---
\begin{frame}{Ba nhóm Design Patterns}
  \begin{columns}[T]
    \begin{column}{0.32\textwidth}
      \begin{block}{\centering Creational}
        \small
        \textit{Tạo object}\\[0.3em]
        Kiểm soát cách khởi tạo đối tượng, giảm dependency vào concrete class
      \end{block}
    \end{column}
    
    \begin{column}{0.32\textwidth}
      \begin{block}{\centering Structural}
        \small
        \textit{Tổ chức cấu trúc}\\[0.3em]
        Kết hợp objects/classes thành cấu trúc lớn hơn mà vẫn linh hoạt
      \end{block}
    \end{column}
    
    \begin{column}{0.32\textwidth}
      \begin{block}{\centering Behavioral}
        \small
        \textit{Phân phối hành vi}\\[0.3em]
        Quản lý giao tiếp và trách nhiệm giữa các objects
      \end{block}
    \end{column}
  \end{columns}
\end{frame}

% --- SLIDE 5: Lưu ý quan trọng ---
\begin{frame}{Lưu ý khi sử dụng Design Patterns}
  \begin{exampleblock}{Dùng đúng lúc}
    \begin{itemize}
      \item Pattern không phải ``silver bullet'' --- dùng sai = over-engineering
      \item Chỉ áp dụng khi gặp vấn đề cụ thể mà pattern giải quyết
      \item Pattern $\neq$ Framework --- pattern là ý tưởng, framework là implementation
    \end{itemize}
  \end{exampleblock}
  
  \begin{alertblock}{Tránh lạm dụng}
    \begin{itemize}
      \item KISS --- Keep It Simple, Stupid
      \item YAGNI --- You Aren't Gonna Need It
      \item Đừng dùng pattern chỉ vì ``nghe hay''
    \end{itemize}
  \end{alertblock}
\end{frame}
